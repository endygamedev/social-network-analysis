\documentclass{spbseu}


\title[]{Учебная практика}
\subtitle{<<Анализ социальных графов>>}
\titlegraphic{\vspace{-1.5cm}\includegraphics[width=8cm,height=4cm]{title}}
\date{}
\author[]{Бронников Егор}


\begin{document}

	\begin{frame}
		\titlepage
	\end{frame}
	
	\begin{frame}
		\tableofcontents
	\end{frame}
	
	\section{Введение}
	
	\begin{frame}
		\frametitle{Введение}
		\tableofcontents[part=1, pausesections]
		Курсовая работа посвящена моделированию оптимизационных задач средствами динамического программирования.
		\vskip5pt
		\begin{block}{Цель исследования}
            \justifying{Обоснование метода динамического программирования при моделировании оптимизационных задач, решение задач аналитически и с помощью программных средств.}
		\end{block}
		\vskip5pt
		\begin{block}{Объект исследования}
			\justifying{Оптимизационные модели, решаемые методом динамического программирования.}
		\end{block}
		\vskip5pt
		\begin{block}{Предмет исследования}
			\justifying{Алгоритм метода динамического программирования.}
		\end{block}
	\end{frame}
	
	\begin{frame}
		\frametitle{Введение}
		\begin{block}{Задачи работы}
			\begin{enumerate}
				\item изучить общий подход динамического программирования;
				\item выявить оптимизационные модели, решаемые методом динамического программирования;
				\item продемонстрировать применение метода динамического программирования при решении оптимизационных задач аналитически;
				\item выполнить программную реализацию некоторых моделей.
			\end{enumerate}
		\end{block}
		\vskip10pt
		\justifying{Результатом работы является программы решения нескольких оптимизационных задач, таких как задача о рюкзаке, задачи о замене оборудования, задача о распределении инвестиций.}
	\end{frame}
    \section{Метод динамического программирования}
    \begin{frame}
        \frametitle{Метод динамического программирования}
		\tableofcontents[part=2]
        \justifying{\textit{Динамическое программирование} является разделом математической науки, в основе которой лежит изучение экстремальный задач управления, планирования и разработка методов их решения, в котором процесс принятия решения разбивается на отдельные этапы.}
        \vskip15pt
        \begin{block}{Общая постановка задачи динамического программирования}
            \justifying{Управляемая физическая система, характеризуется определённым набором параметров. Требуется построить оптимальное решение, на множестве допустимых решений, переводящее систему из начального состояния в конечное состояние, обеспечив целевой функции нужный экстремум.}
        \end{block}
    \end{frame}
	\begin{frame}
        \frametitle{Метод динамического программирования}
        \begin{block}{Принцип оптимальности Беллмана}
            \justifying{Оптимальное поведение обладает тем свойством, что каковы бы ни были исходное состояние и первоначальное решение, последующие решения должны составлять оптимальное поведение относительно состояния, получающегося в результате первоначального решения.}
        \end{block}
        \pause
        \vskip10pt
        \justifying{Рассмотрим функции $B_0(x_0), B_1(x_1), \dots, B_n(x_n)$. Функции $B_i(x_i), i = 0, 1, \dots, n$ представляют собой максимальные значения сумм частных целевых функций $z_{i+1}(x_i, u_{i+1}) + \dots + z_n(x_{n-1},u_n)$, вычисляемые по всем допустимым наборам управлений $(u_{i+1}, \dots, u_n)$.}
        \vskip5pt
        \begin{block}{Принцип оптимальности Беллмана через функциональное уравнение}
            \[B_{i-1}(x_{i-1}) = \max_{u_i}\{z_{i}(x_{i-1}, u_i) + B_i(x_i) \; | \; x_i = f_i(x_{i-1},u_i)\}\]
        \end{block}
    \end{frame}
    \section{Задачи, решаемые методом динамического программирования}
    \begin{frame}
        \frametitle{Задачи, решаемые методом динамического программирования}
		\tableofcontents[part=3]
        \justifying{В основе вычислительных алгоритмов динамического программирования лежит принцип оптимальности Р. Беллмана}
        \vskip15pt
        Данный принцип включает в себя три основных этапа:
        \begin{enumerate}
            \item предварительный этап;
            \item этап условной оптимизации;
            \item этап безусловной оптимизации.
        \end{enumerate}
    \end{frame}
    \section{Задача о замене оборудования}
    \begin{frame}
        \frametitle{Задача о замене оборудования}
		\tableofcontents[part=4]
        \begin{block}{Постановка задачи}
            \justifying{Разработать оптимальную стратегию замены оборудования возраста $k$ лет в плановом периоде продолжительностью $N$ лет, если известны:}
            \vskip5pt
            $\qquad r(t)$ -- стоимость продукции, производимой в течение года на оборудовании возраста $t$ лет $(t = \overline{0,N})$;\\
            $\qquad u(t)$ -- ежегодные расходы, связанные с эксплуатацией оборудования возраста $t$ лет $(t = \overline{0, N})$;\\
            $\qquad s(t)$ -- остаточная стоимость оборудования возраста $t$ лет $(t = \overline{0,N})$;\\
            $\qquad P$ -- стоимость нового оборудования и расходы, связанные с установкой, наладкой и запуском.
        \end{block}
    \end{frame}
    \begin{frame}
        \frametitle{Задача о замене оборудования}
        \begin{block}{Решение}
            \justifying{В начале каждого года имеется две возможности: сохранить оборудование и получить прибыль $r(t)-u(t)$ или заменить его и получить прибыль $s(t)-P+r(0)-u(0)$.}
            \vskip10pt
            \justifying{Прибыль от использования оборудования в последнем $N$-м году планового периода запишется в следующем виде:}\\
            \[F_N(t) = \max \begin{cases} r(t)-u(t) \hspace{1.95cm} - \textup{сохранение}\\ s(t)-P+r(0)-u(0) - \textup{замена} \end{cases}\]
        \end{block}
    \end{frame}
    \begin{frame}
        \frametitle{Задача о замене оборудования}
        \begin{block}{Решение}
            \justifying{А прибыль от использования оборудования в период с $n$-го по $N$-й год:}
            \[F_n(t) = \max \begin{cases} r(t)-u(t)+F_{n+1}(t+1) \hspace{1.35cm} - \textup{сохранение}\\ s(t)-P+r(0)-u(0)+F_{n+1}(1) - \textup{замена}\end{cases} \]
            \justifying{В случае, если оба управления (<<сохранить>> и <<заменить>>) приводят к одной и той же прибыль, то целесообразно выбрать управление <<сохранить>>.}
        \end{block}
    \end{frame}
    \section{Задача о рюкзаке}
    \begin{frame}
        \frametitle{Задача о рюкзаке}
		\tableofcontents[part=5]
        \begin{block}{Постановка задачи}
            \justifying{Пусть дано $N$ предметов, $W$ -- вместимость рюкзака, $w = (w_1, w_2, \dots, w_N)$ -- набор положительных целых весов, $p = (p_1, p_2, \dots, p_N)$ -- соответствующий ему набор положительных целых стоимостей. Нужно найти набор бинарных величин $B = (b_1, b_2, \dots, b_N)$, который обеспечит максимальную стоимость рюкзака, где $b_i = 1$, если предмет $i$ велючён в набор, $b_i = 0$, если предмет $i$ не включён.}
            \[\textup{Целевая функция: }\sum_{i=1}^Nb_ip_i \longrightarrow \max\]
            \begin{nscenter}Ограничения:\end{nscenter}
            \[\sum_{i=1}^Nb_iw_i \leq W\]
            \[b_i \in \{0, 1\} \; \forall i \in \{1, \dots, N\}\]
        \end{block}
    \end{frame}
    \begin{frame}
        \frametitle{Задач о рюкзаке}
        \begin{block}{Решение}
            \justifying{Пусть $F(k,s)$ есть максимальная стоимость предметов, которые можно уложить в рюкзак вместиости $s$, елси можно использовать только первые $k$ предметов, то есть $(n_1, n_2, \dots, n_k)$ назовём этот набор допустимых предметов для $F(k,s)$. Исходя из этого соображения получается, что $F(k,0)=0$ и $F(0,s)=0$.}
        \end{block}
        \vskip10pt
        Возможные варианты для $F(k,s)$:
        \[F(k,s) = \max(F(k-1,s), F(k-1,s-w_k)+p_k)\]
        \justifying{Стоимость искомого набора равна $F(N,W)$, так как нужно найти максимальную стоимость рюкзака, где все предметы допустимы и вместимость рюкзака $W$.}
    \end{frame}
    \section{Задача о распределении инвестиций}
    \begin{frame}
        \frametitle{Задача о распределении инвестиций}
		\tableofcontents[part=6]
        \begin{block}{Постановка задачи}
            \justifying{В производственное объединение входят $N$ предприятий $\textup{П}_1, \textup{П}_2, \dots, \textup{П}_N$. Руководство объединения решило инвестировать в свои предприятия $M$ условных единиц в общей сумме. Проведённые маркетинговые исследования прогнозируют величину ожидаемой прибыли каждого из предприятий в зависимости от объёма инвестированных средств. Требуется найти такое распределение инвестиций между предприятиями, которое обеспечило бы максимум суммарной ожидаемой прибыли.}
        \end{block}
    \end{frame}
    \begin{frame}
        \frametitle{Задача о распределении инвестиций}
        \begin{block}{Решение}
            \justifying{Пусть $S$ -- количество средств, имеющихся в наличи перед шагом. Выигрыш $\textup{П}_i(x_i)$ на $i$-ом шаге -- это прибыль, которую приносит $i$-ое предприятие при инвестировании в него средств $x_i$. Если через выигрыш в целом обозначить общую прибыль $W$, то он будет выглядеть следующим образом:}
            \[W(S)=\textup{П}_1(x_1) + \textup{П}_2(x_2) + \dots + \textup{П}_N(x_N)\]
            \pause
            \justifying{Согласно принципу оптимальности Беллмана, управления на каждом шаге нужно выбирать так, чтобы оптимальной была сумма выигрышей на всех оставшихся до конца процесса шагах, включая выигрыш на данном шаге. Тогда функциональное уравнение Беллмана примет вид:}
            \[W_i(S) = \max_{x \leq S}\{\textup{П}_i(x) + W_{i+1}(S-x)\}\]
        \end{block}
    \end{frame}
    \section{Заключение}
    \begin{frame}
        \frametitle{Заключение}
		\tableofcontents[part=7]
        \justifying{В ходе выполнения курсовой работы достигнуты следующие результаты:}
        \begin{enumerate}
            \item Проанализирован общей подход динамического программирования к решению некоторых типов производственных оптимизационных задач.
            \item Приводится аналитическое решение представленных задач с помощью метода динамического программирования.
            \item Выполнена реализация алгоритма представленных оптимизационных задач.
        \end{enumerate}
    \end{frame}
\end{document}
